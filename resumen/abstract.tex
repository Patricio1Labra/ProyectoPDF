\chapter*{Abstract\markboth{Abstract}{Abstract}}

This thesis aims to address the identified need in the Automated Production Systems Laboratory, known as CIMUBB, where training in automation is provided through robots and computers. The limited access to the laboratory and the difficulty in conducting classes during exceptional situations, such as the pandemic, have generated the need to find a solution that allows for remote teaching.

The proposal in this report is to develop a simulator in Unity, a game engine, that replicates the experience of working with the Scorbot robotic arms used in the laboratory. This simulator will enable students to interact virtually with the robotic arms, performing practical exercises, programming, and tests, similarly to how they would in the actual laboratory. In this way, the aim is to preserve the quality of teaching and overcome the limitations of physical access to the laboratory.

The project will focus on recreating the Scorbot robotic arms, including their functionality and behavior. Various features and tools will be implemented to facilitate the understanding and practice of automation concepts, allowing students to engage in theoretical and practical activities from anywhere.

The expected outcome is for this simulator to become a valuable tool for students and teachers at the CIMUBB laboratory, enhancing the quality of education in automated production systems and providing the possibility of conducting courses virtually or as a complement to in-person classes.


\renewcommand{\keywords}{\textbf{\emph{Keywords ---~}}}
\keywords{Simulator, Scorbot Robotic Arms, Unity, Automated Production Systems, Virtual Teaching, Automation, Robotics.}