\chapter*{Abstract\markboth{Abstract}{Abstract}}

The thesis aims to address the identified need in the Automated Production Systems Laboratory, known as CIMUBB, where training in automation using robots and computers is provided. The limited access to the laboratory and the difficulty of conducting classes during exceptional situations, such as the pandemic, have created the need to find a solution that allows for teaching without physical presence.

The proposal in this report is to develop a simulator in Unity, a game engine, that replicates the experience of working with Scorbot robotic arms used in the laboratory. This simulator allows students to interact virtually with the robotic arms, performing practical exercises, programming, and tests, similar to how they would in the real laboratory. In this way, the aim is to preserve the quality of teaching and overcome the limitations of physical access to the laboratory.

The project's approach is based on the recreation of Scorbot robotic arms, including their functionality and behavior. Various features and tools are implemented to facilitate the understanding and practice of automation concepts, allowing students to engage in theoretical and practical activities from anywhere.

The expected outcome is for this simulator to become a valuable tool for students and professors at the CIMUBB laboratory, improving the quality of teaching in automated production systems and providing the possibility of conducting courses virtually or as a complement to in-person classes.

\renewcommand{\keywords}{\textbf{\emph{Keywords ---~}}}
\keywords{Simulator, Scorbot Robotic Arms, Unity, Automated Production Systems, Virtual Teaching, Automation, Robotics.}