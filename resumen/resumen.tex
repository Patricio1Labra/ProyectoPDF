
\chapter*{Resumen\markboth{Resumen}{Resumen}}

La tesis tiene como objetivo abordar la necesidad identificada en el Laboratorio de Sistemas Automatizados de Producción, conocido como CIMUBB, donde se imparte formación en automatización mediante robots y computadoras. La limitación de acceso al laboratorio y la dificultad para impartir clases durante situaciones excepcionales, como la pandemia, han generado la necesidad de encontrar una solución que permita brindar enseñanza sin presencia física.

La propuesta en este informe, consiste en desarrollar un simulador en Unity, un motor de videojuegos, que reproduzca la experiencia de trabajar con los brazos robóticos Scorbot utilizados en el laboratorio. Este simulador permite a los estudiantes interactuar virtualmente con los brazos robóticos, realizando ejercicios prácticos, programación y pruebas, de manera similar a como lo harían en el laboratorio real. De esta forma, se busca preservar la calidad de la enseñanza y superar las limitaciones de acceso físico al laboratorio.

El enfoque del proyecto se basa en la recreación de los brazos robóticos Scorbot, incluyendo su funcionamiento y comportamiento. Se implementan diversas funcionalidades y herramientas que faciliten la comprensión y práctica de los conceptos de automatización, permitiendo que los estudiantes realicen actividades teóricas y prácticas desde cualquier lugar.

El resultado esperado es que este simulador se convierta en una valiosa herramienta para estudiantes y profesores del laboratorio CIMUBB, mejorando la calidad de la enseñanza en sistemas automatizados de producción y brindando la posibilidad de impartir cursos de forma virtual o complementaria a las clases presenciales.


\renewcommand{\keywords}{\textbf{\emph{Palabras Clave ---~}}}
\keywords{Simulador, Brazos Robóticos Scorbot, Unity, Sistemas Automatizados de Producción, Enseñanza Virtual, Automatización, Robótica.}