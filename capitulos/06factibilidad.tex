En el capítulo actual se mostrará la factibilidad que tiene el proyecto de ser realizado.

\section{Factibilidad Técnica}

Requerimiento para el desarrollo:
\begin{table}[h!]
\centering
\begin{tabular}{ | p{0.13\linewidth} | p{0.2\linewidth} | p{0.25\linewidth} | p{0.3\linewidth} |}
\noalign{\hrule height 2pt}
Requisitos

mínimos & Windows & macOS & Linux \\ 
\noalign{\hrule height 2pt}

Versión 

sistema 

operativo & 
Windows 7 

(Service Pack 1+),

Windows 10 y 

Windows 11 & 
High Sierra 10.13+ &
Ubuntu 20.04, Ubuntu 18.04, y CentOS 7
 \\
\hline

Procesador & 
Arquitectura x86 o

x64 con soporte de

instrucción SSE2 & 
Arquitectura x64 con

soporte de instrucción

SSE2 para CPU Intel,

Apple M1 o superior

para CPU Apple & 
Arquitectura x64 con soporte

de instrucción SSE2
 \\
\hline

API Gráfico & 
GPU compatible

con DirectX 

versión 10, 11 o 12 & 
GPU Intel, AMD o Apple compatible con Metal &
GPU compatible con

OpenGL 3.2+ o Vulkan
 \\
\hline

\end{tabular}
\caption{Requerimientos de desarrollo en computadores}
\end{table}

\clearpage
Factibilidad Técnica para Desarrollo Unity en un Notebook con Ryzen 7 5700U, 16GB RAM y Ubuntu:

\begin{enumerate}[label=\arabic*.]
\item \textbf{Compatibilidad con Sistema Operativo:}
\begin{itemize}
\item \textbf{Factible:} El sistema operativo Ubuntu 20.04 es compatible con los requisitos mínimos para el desarrollo en Unity. Además, la arquitectura x64 del procesador Ryzen 7 5700U es compatible.
\end{itemize}

\item \textbf{Procesador:}
\begin{itemize}
\item \textbf{Factible:} El Ryzen 7 5700U es un procesador x64 con soporte para instrucciones SSE2, cumpliendo así con los requisitos mínimos de Unity.
\end{itemize}

\item \textbf{API Gráfico:}
\begin{itemize}
\item \textbf{Factible:} Ubuntu 20.04 es compatible con OpenGL 3.2+, y el Ryzen 7 5700U tiene una GPU integrada que es capaz de soportar esta versión de OpenGL.
\end{itemize}

\item \textbf{Memoria RAM:}
\begin{itemize}
\item \textbf{Factible:} Con 16GB de RAM, el sistema cumple con los requisitos mínimos, lo que proporcionará suficiente memoria para el desarrollo en Unity.
\end{itemize}

\item \textbf{Conexión a Internet:}
\begin{itemize}
\item \textbf{Factible:} Dado que algunos recursos y actualizaciones de Unity pueden requerir una conexión a Internet, es recomendable tener acceso a una conexión estable.
\end{itemize}

\item \textbf{Espacio en Disco:}
\begin{itemize}
\item \textbf{Factible:} Con 400GB disponibles en un disco de 500GB, cumples con el requisito mínimo de 15GB libres que necesita Unity. Es completamente factible.
\end{itemize}

\item \textbf{Compatibilidad con Controladores Gráficos:}
\begin{itemize}
\item \textbf{Factible:} Los controladores gráficos usados en Vulkan son mesa v23, siendo así compatibles con Unity.
\end{itemize}
\end{enumerate}

En resumen, después de analizar todos los aspectos técnicos, se determina que el proyecto es factible. Las especificaciones del notebook cumplen con los requisitos mínimos de Unity, y los recursos necesarios están disponibles para el desarrollo sin necesidad de hardware adicional significativo.

\clearpage

\section{Factibilidad Operativa}

\begin{enumerate}
    \item \textbf{Recursos Humanos:}
    Como desarrollador del simulador con experiencia en programación y Unity, los recursos humanos necesarios ya están disponibles internamente. La capacitación para los usuarios se limita a conocimientos básicos sobre el uso de un dispositivo.

    \item \textbf{Recursos Físicos:}
    Con los datos que entrega la cámara de diputados en 2022\cite{InformeCamara}, se puede asumir que de los 3.89 millones de usuarios de internet fija en la zona residencial, debe tener al menos un computador, por lo cual se dice que al menos 3.89 millones de personas serian potenciales usuarios.

    \item \textbf{Tecnología y Sistemas de Información:}
    La restricción de hardware para Unity (computadoras desde 2012 y celulares Android desde 2011) se considera aceptable y probablemente cubra una gran parte de los dispositivos en uso actualmente.

    \item \textbf{Cumplimiento Normativo:}
    Dado que el simulador no almacena datos, no hay problemas de privacidad y seguridad que considerar.

    \item \textbf{Experiencia del Usuario:}
    La interfaz del simulador se diseñó de manera simple para facilitar su uso.

\end{enumerate}

La factibilidad operativa del simulador de brazos robóticos en Unity es sólida. Con recursos humanos disponibles, un amplio alcance potencial, requisitos tecnológicos razonables y la ausencia de procesos operativos complejos, el simulador parece ser una herramienta eficiente y accesible para entornos educativos.

\section{Impacto Positivo:}

\begin{enumerate}
    \item \textbf{Experimentación sin riesgos:} Los entornos virtuales en Unity permiten a los investigadores y estudiantes realizar experimentos y pruebas sin riesgos ni costos asociados con el mundo real. Esto es especialmente útil en campos como la investigación médica, química o física, donde los experimentos pueden ser costosos o peligrosos.
    
    \item \textbf{Acceso universal:} Los entornos virtuales pueden ser accesibles desde cualquier lugar, lo que facilita la colaboración y el acceso a recursos educativos o de investigación para personas de todo el mundo.
    
    \item \textbf{Aprendizaje interactivo:} Para fines educativos, un entorno virtual en Unity puede proporcionar una experiencia de aprendizaje más interactiva y atractiva. Puede mejorar la comprensión de conceptos complejos al permitir a los estudiantes interactuar directamente con modelos y simulaciones.
    
    \item \textbf{Personalización:} Los entornos virtuales pueden adaptarse para satisfacer las necesidades específicas del laboratorio o del programa educativo. Esto permite una mayor personalización y adaptabilidad a diferentes objetivos de investigación o de aprendizaje.
    
    \item \textbf{Reducción de costos:} Al utilizar entornos virtuales, se pueden reducir los costos asociados con la compra de equipos especializados, materiales y mantenimiento de laboratorios físicos.
\end{enumerate}

\section{Impacto Negativo:}

\begin{enumerate}
    \item \textbf{Limitaciones de simulación:} Aunque los entornos virtuales son poderosos, no siempre pueden replicar completamente la complejidad del mundo real. Pueden haber limitaciones en la simulación de ciertos fenómenos o situaciones.
    
    \item \textbf{Requisitos técnicos:} La implementación y el uso de entornos virtuales en Unity pueden requerir habilidades técnicas y recursos significativos. Esto podría ser un desafío para algunos laboratorios que no tienen acceso a personal técnico o hardware adecuado.
    
    \item \textbf{Aislamiento:} Dependiendo de cómo se implemente, la tecnología virtual puede llevar al aislamiento del mundo real. Los estudiantes o investigadores pueden perder la experiencia táctil y la interacción directa con los objetos y fenómenos que están estudiando.
    
\end{enumerate}

\section{Factibilidad Económica}

\textbf{Costo de personal}

Dado que el proyecto de tesis se desarrolla dentro de un entorno académico y no implica la contratación ni la compensación económica adicional para el personal involucrado, no se incluye un ítem de gasto en recursos humanos. La naturaleza educativa y formativa del proyecto permite que los participantes, principalmente los estudiantes responsables de la tesis, contribuyan sin generar desembolsos económicos adicionales para el proyecto. La ausencia de este gasto fortalece la viabilidad económica del proyecto en el ámbito académico.

\textbf{Costo de desarrollo}

El gasto asociado al desarrollo del proyecto es asumido por un estudiante en proceso de titulación, lo que implica que el costo directo asignado al estudiante es de 0 CLP. Sin embargo, con el propósito de realizar una evaluación económica, se emplea el promedio de la tarifa por hora de un ingeniero civil informático, fijada en 6000 pesos chilenos. Esta estimación se utiliza como una medida representativa del valor del tiempo y la contribución del estudiante al proyecto, a pesar de que no se traduzca en un desembolso financiero directo en este contexto académico.

\begin{table}[ht]
    \centering
    \begin{tabular}{|l|c|c|c|c|}
        \hline
        \textbf{Fase} & \textbf{Porcentaje} & \textbf{Horas} & \textbf{Costo por Hora} & \textbf{Costo Total} \\
        \hline
        Diseño & 10\% & 58.5 & 6000 & 351,000 \\
        \hline
        Desarrollo & 45\% & 263.25 & 6000 & 1,579,500 \\
        \hline
        Implementación & 20\% & 117 & 6000 & 702,000 \\
        \hline
        Pruebas & 15\% & 87.75 & 6000 & 526,500 \\
        \hline
        \multicolumn{4}{|r|}{\textbf{Costo Total de Desarrollo}} & \textbf{3,159,000} CLP \\
        \hline
    \end{tabular}
    \caption{Costo de Desarrollo del Proyecto en Unity}
    \label{tab:costo-desarrollo}
\end{table}


\textbf{Costo de hardware}
La ausencia de inversión en hardware se justifica por el hecho de que la aplicación se ejecutará de manera eficiente en los computadores personales de los estudiantes, eliminando así la necesidad de adquirir equipamiento adicional. Este enfoque estratégico garantiza que los recursos tecnológicos existentes sean plenamente utilizados, minimizando cualquier requerimiento financiero relacionado con la infraestructura de hardware del proyecto.

\begin{table}[ht]
    \centering
    \label{tab:financial-evaluation}
    \begin{tabular}{ | p{0.18\linewidth} | c | c | c | c | c | c | }
        \hline
        \textbf{Año} & \textbf{0} & \textbf{1} & \textbf{2} & \textbf{3} & \textbf{4} & \textbf{5} \\
        \hline
        
        Desarrollo & -3,159,000 & - & - & - & - & - \\
        \hline
        Utilidad & - & 165,000 & 330,000 & 495,000 & 660,000 & 825,000 \\
        \hline
        Impuesto & - & 28,050 & 56,100 & 84,150 & 112,200 & 140,250 \\
        \hline
        Utilidad después 
        
        de impuesto & - & 136,950 & 273,900 & 410,850 & 547,800 & 684,750 \\
        \hline
        Utilidad del 
        
        periodo & -3,159,000 & -3,022,050 & -2,748,150 & -2,337,300 & -1,789,500 & -1,104,750 \\
        \hline
    \end{tabular}
    \caption{Evaluación Financiera del Proyecto}
\end{table}

\begin{enumerate}[label=\arabic*.]
    \item \textbf{Desarrollo:} Este campo representa los costos asociados con la fase de desarrollo del proyecto. En el año 0, se muestra el monto de inversión inicial necesario para el desarrollo del proyecto. En el ejemplo proporcionado, este valor es de -3,159,000 CLP, lo que indica un gasto.
    
    \item \textbf{Utilidad:} Representa la ganancia generada por el proyecto en cada año después de la deducción de los costos y gastos, pero antes de impuestos. En este contexto, la "Utilidad" refleja la estimación de reducción de costos asociados con el mantenimiento del brazo robótico. Inicialmente, el costo de mantenimiento es de 165,000 CLP por mes. Gracias a la implementación del software, se estima que la eficiencia aumentará, lo que resultará en la reducción de un mes de mantenimiento por año. Por lo tanto, la "Utilidad" refleja tanto los ahorros en costos de mantenimiento como cualquier otra ganancia generada por el proyecto, sin considerar impuestos en esta etapa del análisis.
    
    \item \textbf{Utilidad del periodo:} Representa la ganancia neta después de impuestos y es el resultado final de restar el impuesto de la utilidad antes de impuestos. Este valor muestra la ganancia neta real que el proyecto genera después de cumplir con las obligaciones tributarias. En el ejemplo, los valores de "Utilidad del periodo" son negativos en los primeros años debido a los costos iniciales y el impacto del impuesto.
\end{enumerate}

\clearpage
\section{Conclusión de la factibilidad}

En conclusión, después de analizar la factibilidad técnica, operativa y económica del proyecto de simulador de brazos robóticos en Unity, se puede afirmar que el proyecto es factible y presenta un potencial impacto positivo en diversas áreas.

Desde el punto de vista técnico, los requisitos de hardware y software necesarios para el desarrollo en Unity son cumplidos por el equipo de desarrollo propuesto, especialmente en el caso del notebook con Ryzen 7 5700U y Ubuntu. Este análisis asegura que la implementación del proyecto no requiera hardware adicional significativo.

En términos operativos, la disponibilidad de recursos humanos, físicos y tecnológicos es sólida. El proyecto cuenta con un desarrollador con experiencia en programación y Unity, y se proyecta un amplio alcance potencial gracias a la cantidad estimada de usuarios de internet fija en la zona residencial.

El impacto positivo del proyecto se refleja en la posibilidad de realizar experimentos sin riesgos, el acceso universal a la herramienta, el aprendizaje interactivo, la personalización según las necesidades y la reducción de costos asociados con laboratorios físicos.

La evaluación económica muestra que, a pesar de los costos iniciales de desarrollo, el proyecto tiene el potencial de generar utilidades a partir del segundo año. La ausencia de costos de personal y hardware adicionales contribuye a la viabilidad económica del proyecto, y el análisis financiero proyecta una utilidad neta positiva en los años subsiguientes.

En resumen, la factibilidad del proyecto se respalda sólidamente en los aspectos técnico, operativo y económico, lo que sugiere que la implementación del simulador de brazos robóticos en Unity es una propuesta viable y prometedora.