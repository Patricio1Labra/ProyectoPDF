En este capitulo se detallan los objetivos, la metodología que se utiliza, y las tecnologías en la cual se apoya.

\section{Objetivos del proyecto}

\subsection{Objetivo General}
Desarrollar una aplicación en Unity que permita simular el laboratorio
del CIMUBB, con la cual los estudiantes puedan probar el funcionamiento de los
brazos robóticos SCORBOT ER V Plus y el brazo robótico SCORBOT ER IX , intentando replicar con el mayor detalle posible
las estaciones de los brazos en el laboratorio.

\subsection{Objetivos Específicos}
\begin{enumerate}[label=\roman*.-]
\item Examinar los desafíos inherentes de la problemática que presenta el laboratorio.
\item Analizar tecnologías y herramientas con el fin de comprender su funcionamiento y utilizarlas de manera eficaz en la resolución del problema.
\item Desarrollar la aplicación.
\end{enumerate}

\subsection{Actividades}
Las actividades desarrolladas en este proyecto son:
\begin{enumerate}[label=\roman*.-]
\item Examinar los desafíos inherentes de la problemática que presenta el laboratorio.
    \begin{enumerate}[label=\arabic*.-]
    \item Investigación sobre los factores que originan la problemática.
    \end{enumerate}
\item Analizar tecnologías y herramientas con el fin de comprender su funcionamiento y utilizarlas de manera eficaz en la resolución del problema.
    \begin{enumerate}[label=\arabic*.-]
    \item Exploración y análisis de diversas herramientas de programación con el propósito de comprender sus características y funcionalidades.
    \item Investigación para evaluar las posibilidades que ofrecen distintas herramientas de programación y determinar cuál es la más adecuada para abordar el proyecto.
    \end{enumerate}
\item Desarrollar la aplicación
    \begin{enumerate}[label=\arabic*.-]
    \item Investigación sobre el laboratorio.
    \item Diseño de entorno y equipamiento.
    \item Desarrollo de código para dar funciones.
    \item Diseño de interfaz gráfica.
    \item Desarrollo de funcionalidad de la interfaz gráfica.
    \end{enumerate}
\end{enumerate}

\section{Ambiente de ingeniería de software}
\subsection{Metodología de Desarrollo}
La metodología de desarrollo a utilizar a lo largo del proyecto es incremental \cite{Incremental}, pues se debe a que facilita al desarrollo del proyecto y saber si se encuentra por un buen camino, debido a que se le presenta cada incremento al cliente y obtener una retroalimentación del mismo.

\subsection{Tecnologías}
\begin{table}[h!]
\begin{center}
\begin{tabular}{ m{0.15\linewidth} m{0.12\linewidth} m{0.65\linewidth} }
\noalign{\hrule height 2pt}
Nombre & Logo & Descripción \\ 
\noalign{\hrule height 2pt}

C\# & 
\includegraphics[height=0.12\textwidth]{figures/c.png} & 
C\# es uno de los lenguajes de programación diseñados para la infraestructura de lenguaje común . Su sintaxis básica deriva de C / C++ y utiliza el modelo de objetos de la plataforma .NET, similar al de Java, aunque incluye mejoras derivadas de otros lenguajes.
 \\
\hline
\end{tabular}
\caption{Tecnologías}
\end{center}
\end{table}

\clearpage
\subsection{Herramientas de apoyo}
\begin{table}[h!]
\begin{center}
\begin{tabular}{ m{0.15\linewidth} m{0.12\linewidth} m{0.65\linewidth} }
\noalign{\hrule height 2pt}
Nombre & Logo & Descripción \\ 
\noalign{\hrule height 2pt}

Unity & 
\includegraphics[height=0.12\textwidth]{figures/Unity.png} & 
Unity es un motor de desarrollo en tiempo real que te permite crear experiencias interactivas en el Editor de Unity que se utiliza para la creación de videojuegos. Estos se pueden publicar en diversas plataformas como PC, videoconsolas, móviles, etc. Gracias a su flexibilidad es una herramienta que también se usa en diferentes industrias como arquitectura, ingeniería, automotriz y de entretenimiento.
 \\
\hline

Blender & 
\includegraphics[height=0.09\textwidth]{figures/Blender.png} & 
Blender es un programa informático multiplataforma, dedicado especialmente al modelado, iluminación, renderizado, la animación y creación de gráficos tridimensionales. También de composición digital utilizando la técnica procesal de nodos, edición de vídeo, escultura (incluye topología dinámica) y pintura digital.
\\
\hline

Visual Studio Code & 
\includegraphics[height=0.1\textwidth]{figures/VSC.png} & 
Es un editor de código fuente desarrollado por Microsoft para Windows, Linux y macOS. Incluye soporte para la depuración, control integrado de Git, resaltado de sintaxis, finalización inteligente de código, fragmentos y refactorización de código.
\\ 
\hline

GitHub & 
\includegraphics[height=0.1\textwidth]{figures/GitHub.png} & 
Es un servicio en la nube que ayuda a los desarrolladores a almacenar y administrar su código, al igual que llevar un registro y control de cualquier cambio sobre este código.
\\ 
\hline

\end{tabular}
\caption{Herramientas de apoyo}
\end{center}
\end{table}
