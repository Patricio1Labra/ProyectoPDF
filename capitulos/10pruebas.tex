Este capítulo se enfoca en las diferentes pruebas del software desarrollado. Se destaca su importancia para garantizar la calidad y funcionalidad del producto final, asegurando que cumpla con los requisitos establecidos.
\section{Elementos de prueba}
Se realiza una prueba con 4 estudiantes de Ingeniería de Ejecución Electronica que recientemente realizaron clases utilizando los equipos del laboratorio y el profesor Luis.
Estos realizan las pruebas del programa y rellenan una encuesta.

\section{Detalle de las pruebas solicitadas por el desarrollador}
El usuario de prueba utiliza el programa sin algún condicionamiento, este solo conoce que el programa es para ''recrear'' el laboratorio.
Después de utilizar el programa, rellena la siguiente encuesta
\begin{itemize}
\item ¿Es fácil usar la botonera del robot?
\item ¿Es fácil realizar los cambios de cámara?
\item ¿Es fácil cambiar de brazo robótico?
\item El menu inicial ¿Es fácil de entender y sencillo de usar?
\item La interfaz dentro de la simulación ¿Es fácil de entender y sencillo de usar?
\end{itemize}

Esta encuesta se realizo mediante Google Forms, las cuales también dan la opción de redactar para dar comentarios.

\clearpage
\section{Respuestas}

\subsection*{¿Es fácil usar la botonera del robot?}
\begin{table}[ht!]
\centering
\begin{tabular}{| p{0.2\linewidth} | p{0.3\linewidth} | p{0.4\linewidth} |}
\noalign{\hrule height 2pt}
\textbf{Nombre} & \textbf{Respuesta} & \textbf{Comentario} \\
\noalign{\hrule height 2pt}
Luis & Si & Falta el modo XYZ, eso facilitaría poder tomar objetos, también faltan otras funciones\\
\hline
Tomas & Si & Fácil de usar para el operario\\
\hline
Carlos & Si & Si es fácil de usar ya que es interactiva \\
\hline
Sebastian & Si & es similar a la del laboratorio \\
\hline
Fabian & Si & Con el uso adecuado y entreno respectivo si se hace ''fácil''\\
\hline
\end{tabular}
\caption{Facilidad de uso de la botonera del robot}
\end{table}

\begin{figure}[ht]
\centering
\begin{tikzpicture}
\pie[text=legend, color={green!70, red!70}, radius=1.5]
{100/¡Sí, 0/No}
\end{tikzpicture}
\caption{Facilidad de uso de la botonera del robot}
\label{fig:usobotonera}
\end{figure}

\subsection*{¿Es fácil realizar los cambios de cámara?}
\begin{table}[ht!]
\centering
\begin{tabular}{| p{0.2\linewidth} | p{0.3\linewidth} | p{0.4\linewidth} |}
\noalign{\hrule height 2pt}
\textbf{Nombre} & \textbf{Respuesta} & \textbf{Comentario} \\
\noalign{\hrule height 2pt}
Luis & Si & El listado de cámaras confunde. Quizás utilizar un nombre descriptivo pueda ser util. Quizás considerar utilizar el cambio de vistas como en los juegos arcade\\
\hline
Tomas & Si & \\
\hline
Carlos & No & Mejor anclaje de cámara \\
\hline
Sebastian & Si &  \\
\hline
Fabian & No & \\
\hline
\end{tabular}
\caption{Facilidad de uso del cambio de cámara}
\end{table}

\begin{figure}[ht]
\centering
\begin{tikzpicture}
\pie[text=legend, color={green!70, red!70}, radius=1.5]
{60/Sí, 40/No}
\end{tikzpicture}
\caption{Facilidad de uso del cambio de cámara}
\label{fig:usocamara}
\end{figure}

\subsection*{¿Es fácil cambiar de brazo robótico?}
\begin{table}[ht!]
\centering
\begin{tabular}{| p{0.2\linewidth} | p{0.3\linewidth} | p{0.4\linewidth} |}
\noalign{\hrule height 2pt}
\textbf{Nombre} & \textbf{Respuesta} & \textbf{Comentario} \\
\noalign{\hrule height 2pt}
Luis & Si & \\
\hline
Tomas & Si & \\
\hline
Carlos & Si & \\
\hline
Sebastian & Si & \\
\hline
Fabian & Si & \\
\hline
\end{tabular}
\caption{Facilidad de uso del cambio de brazo robótico}
\end{table}

\begin{figure}[ht]
\centering
\begin{tikzpicture}
\pie[text=legend, color={green!70, red!70}, radius=1.5]
{100/Sí, 0/No}
\end{tikzpicture}
\caption{Facilidad de uso del cambio de brazo robótico}
\label{fig:usobrazo}
\end{figure}

\clearpage
\subsection*{El menú inicial ¿Es fácil de entender y sencillo de usar?}
\begin{table}[ht!]
\centering
\begin{tabular}{| p{0.2\linewidth} | p{0.3\linewidth} | p{0.4\linewidth} |}
\noalign{\hrule height 2pt}
\textbf{Nombre} & \textbf{Respuesta} & \textbf{Comentario} \\
\noalign{\hrule height 2pt}
Luis & Si & \\
\hline
Tomas & Si & \\
\hline
Carlos & Si & Si es fácil de acuerdo con la guía \\
\hline
Sebastian & Si & \\
\hline
Fabian & No & Hay que saber como mover el brazo, y obviamente saber lo que se está haciendo, y para ello se requiere un entreno decente\\
\hline
\end{tabular}
\caption{Facilidad de uso del menu}
\end{table}

\begin{figure}[ht]
\centering
\begin{tikzpicture}
\pie[text=legend, color={green!70, red!70}, radius=1.5]
{80/Sí, 20/No}
\end{tikzpicture}
\caption{Facilidad de uso del menu}
\label{fig:usomenu}
\end{figure}

\subsection*{La interfaz dentro de la simulación ¿Es fácil de entender y sencillo de usar?}
\begin{table}[ht!]
\centering
\begin{tabular}{| p{0.2\linewidth} | p{0.3\linewidth} | p{0.4\linewidth} |}
\noalign{\hrule height 2pt}
\textbf{Nombre} & \textbf{Respuesta} & \textbf{Comentario} \\
\noalign{\hrule height 2pt}
Luis & No & \\
\hline
Tomas & Si & \\
\hline
Carlos & Si & \\
\hline
Sebastian & Si & \\
\hline
Fabian & Si & \\
\hline
\end{tabular}
\caption{Facilidad de uso del menu dentro de la simulación}
\end{table}

\clearpage
\begin{figure}[ht]
\centering
\begin{tikzpicture}
\pie[text=legend, color={green!70, red!70}, radius=1.5]
{80/Sí, 20/No}
\end{tikzpicture}
\caption{Facilidad de uso del menu dentro de la simulación}
\label{fig:usodentro}
\end{figure}

\section{Conclusiones de prueba del desarrollador}
Se puede analizar que mayormente el programa cumple la funcionalidad de ser intuitivo, básicamente cubre las expectativas que se esperaba.
Por otra parte, existen comentarios de mejora como la visual de la cámara.
Por ultimo, el comentario de Fabian que dice que ''hay que tener conocimiento para saber lo que se hace'', indica la complejidad de operación que se tiene con el brazo robótico, aunque para replicar el funcionamiento que tiene en el laboratorio, no se puede hacer de manera más simple.

\section{Prueba del profesor}
El profesor, interesado en evaluar la aplicación desarrollada, ha solicitado a uno de sus alumnos, específicamente a Juan Henríquez de la carrera Ingeniería Civil Informática ,que lleve a cabo la prueba correspondiente. 

\subsection*{Respuesta de la prueba}
La operación de la cámara resulta un tanto torpe.

Se sugiere fijar los brazos para mejorar su estabilidad.

Además, se propone incorporar un sistema visual que indique la posición de cada brazo o establecer un sistema de referencia en la mesa para clarificar la asignación de ejes a cada botón.

En relación al menú, se observa que la resolución no se ajusta a 1080p, y además, carece de utilidad al no mostrar las teclas asignadas. Sería conveniente mejorar la visualización del menú, asegurando su adaptabilidad a la resolución mencionada y proporcionando información clara sobre las teclas asignadas.

La identificación y accesibilidad al botón de abrir/cerrar la garra resulta problemática y debería ser abordada para facilitar su localización y uso eficiente.

En cuanto al sistema de selección de robots, se sugiere optimizarlo para ofrecer únicamente las opciones de cámara correspondientes a la posición del robot seleccionado. Esto ayudaría a evitar maniobrar inadvertidamente con otros robots.

Se destaca que el robot 5 eje z y 9 presentan un comportamiento inesperado al presionar el botón "1".