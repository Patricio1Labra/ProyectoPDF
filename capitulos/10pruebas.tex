Este capítulo se enfoca en las diferentes pruebas del software desarrollado. Se destaca su importancia para garantizar la calidad y funcionalidad del producto final, asegurando que cumpla con los requisitos establecidos.
\section{Elementos de prueba}
Se realiza una prueba con estudiantes de Ingeniería de Ejecución Electronica que recientemente realizaron clases utilizando los equipos del laboratorio.
Estos realizan la prueba del programa y rellenan una encuesta.

\section{Detalle de las pruebas}
El usuario de prueba utiliza el programa sin algún condicionamiento, este solo conoce que el programa es para ''recrear'' el laboratorio.
Despues de utilizar el programa, rellena la siguiente encuesta
\begin{itemize}
    \item ¿Es fácil usar la botonera del robot?
    \item ¿Es fácil realizar los cambios de cámara?
    \item ¿Es fácil cambiar de brazo robótico?
    \item El menu inicial ¿Es fácil de entender y sencillo de usar?
    \item La interfaz dentro de la simulación ¿Es fácil de entender y sencillo de usar?
\end{itemize}

Esta encuesta se realizo mediante Google Forms, las cuales también dan la opción de redactar para dar comentarios.

\clearpage
\section{Respuestas}
\begin{table}[ht!]
\centering
\begin{tabular}{| p{0.2\linewidth} | p{0.3\linewidth} | p{0.4\linewidth} |}
\noalign{\hrule height 2pt}
\textbf{Nombre} & \textbf{¿Es fácil usar la botonera del robot?} & \textbf{Comentario} \\
\noalign{\hrule height 2pt}
Tomas & Si & Fácil de usar para el operario\\
\hline
Carlos & Si & Si es fácil de usar ya que es interactiva \\
\hline
Sebastian & Si & es similar a la del laboratorio \\
\hline
Fabian & Si & Con el uso adecuado y entreno respectivo si se hace ''fácil''\\
\hline
\end{tabular}
\caption{Facilidad de uso de la botonera del robot}
\end{table}

\begin{table}[ht!]
\centering
\begin{tabular}{| p{0.2\linewidth} | p{0.3\linewidth} | p{0.4\linewidth} |}
\noalign{\hrule height 2pt}
\textbf{Nombre} & \textbf{¿Es fácil realizar los cambios de cámara?} & \textbf{Comentario} \\
\noalign{\hrule height 2pt}
Tomas & Si & \\
\hline
Carlos & No & Mejor anclaje de cámara \\
\hline
Sebastian & Si &  \\
\hline
Fabian & No & \\
\hline
\end{tabular}
\caption{Facilidad de uso del cambio de cámara}
\end{table}

\begin{table}[ht!]
\centering
\begin{tabular}{| p{0.2\linewidth} | p{0.3\linewidth} | p{0.4\linewidth} |}
\noalign{\hrule height 2pt}
\textbf{Nombre} & \textbf{¿Es fácil cambiar el brazo robótico?} & \textbf{Comentario} \\
\noalign{\hrule height 2pt}
Tomas & Si & \\
\hline
Carlos & Si & \\
\hline
Sebastian & Si & \\
\hline
Fabian & Si & \\
\hline
\end{tabular}
\caption{Facilidad de uso del cambio de brazo robótico}
\end{table}
\clearpage

\begin{table}[ht!]
\centering
\begin{tabular}{| p{0.2\linewidth} | p{0.3\linewidth} | p{0.4\linewidth} |}
\noalign{\hrule height 2pt}
\textbf{Nombre} & \textbf{El menu inicial ¿Es fácil de entender y sencillo de usar?} & \textbf{Comentario} \\
\noalign{\hrule height 2pt}
Tomas & Si & \\
\hline
Carlos & Si & Si es fácil de acuerdo con la guía \\
\hline
Sebastian & Si & \\
\hline
Fabian & No & Hay que saber como mover el brazo, y obviamente saber lo que se está haciendo, y para ello se requiere un entreno decente\\
\hline
\end{tabular}
\caption{Facilidad de uso del menu}
\end{table}

\begin{table}[ht!]
\centering
\begin{tabular}{| p{0.2\linewidth} | p{0.3\linewidth} | p{0.4\linewidth} |}
\noalign{\hrule height 2pt}
\textbf{Nombre} & \textbf{La interfaz dentro de la simulación ¿Es fácil de entender y sencillo de usar?} & \textbf{Comentario} \\
\noalign{\hrule height 2pt}
Tomas & Si & \\
\hline
Carlos & Si & \\
\hline
Sebastian & Si & \\
\hline
Fabian & Si & \\
\hline
\end{tabular}
\caption{Facilidad de uso del menu dentro de la simulación}
\end{table}


\section{Conclusiones de prueba}
Se puede analizar que mayormente el programa cumple la funcionalidad de ser intuitivo, básicamente cubre las expectativas que se esperaba.
Por otra parte, existen comentarios de mejora como la visual de la cámara.
Por ultimo, el comentario de Fabian que dice que ''hay que tener conocimiento para saber lo que se hace'', indica la complejidad de operación que se tiene con el brazo robótico, aunque para replicar el funcionamiento que tiene en el laboratorio, no se puede hacer de manera más simple.