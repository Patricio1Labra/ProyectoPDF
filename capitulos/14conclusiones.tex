A lo largo de este proyecto, he podido comprender la importancia fundamental de mejorar la educación a través de la simulación \cite{Simulacion}. Aunque la simulación no puede replicar completamente la experiencia real, he observado que puede acercarse lo suficiente como para tener un impacto significativo en el proceso de aprendizaje. La capacidad de crear situaciones virtualmente similares a la realidad resulta especialmente relevante para eliminar barreras educativas, proporcionando alternativas accesibles y asequibles para personas con diversas limitaciones, como problemas de movilidad, foto-sensibilidad, discapacidad y otros obstáculos.

El enfoque de la simulación va más allá de simplemente recrear situaciones en un entorno virtual. Su verdadero potencial radica en la creación de un ambiente inclusivo donde todos puedan acceder a oportunidades educativas de manera efectiva, independientemente de sus capacidades físicas o cognitivas. Al adoptar esta herramienta en el ámbito educativo, se abren puertas para aquellas personas que, de otro modo, enfrentarían dificultades para participar en experiencias de aprendizaje costosas o restringidas.

En resumen, la simulación en la educación tiene el poder de promover una sociedad más equitativa e inclusiva, permitiendo que cada individuo alcance su máximo potencial. Aunque es un campo en desarrollo, espero que esta tesis inspire a educadores, instituciones académicas y desarrolladores tecnológicos a utilizar la simulación como una herramienta transformadora en la búsqueda de una educación para todos, sin barreras ni limitaciones. Al hacerlo, construiremos un futuro en el que la adquisición de conocimientos y habilidades sea una posibilidad accesible para cada persona, contribuyendo así al avance y prosperidad de la sociedad en su conjunto.