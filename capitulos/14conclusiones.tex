A lo largo de este proyecto, he podido comprender la importancia fundamental de mejorar la educación a través de la simulación \cite{Simulacion}. Aunque la simulación no puede replicar completamente la experiencia real, he observado que puede acercarse lo suficiente como para tener un impacto significativo en el proceso de aprendizaje. La capacidad de crear situaciones virtualmente similares a la realidad resulta especialmente relevante para eliminar barreras educativas, proporcionando alternativas accesibles y asequibles para personas con diversas limitaciones, como problemas de movilidad, foto-sensibilidad, discapacidad y otros obstáculos.

El enfoque de la simulación va más allá de simplemente recrear situaciones en un entorno virtual. Su verdadero potencial radica en la creación de un ambiente inclusivo donde todos puedan acceder a oportunidades educativas de manera efectiva, independientemente de sus capacidades físicas o cognitivas. Al adoptar esta herramienta en el ámbito educativo, se abren puertas para aquellas personas que, de otro modo, enfrentarían dificultades para participar en experiencias de aprendizaje costosas o restringidas.

En resumen, la simulación en la educación tiene el poder de promover una sociedad más equitativa e inclusiva, permitiendo que cada individuo alcance su máximo potencial. Aunque es un campo en desarrollo, espero que esta tesis inspire a educadores, instituciones académicas y desarrolladores tecnológicos a utilizar la simulación como una herramienta transformadora en la búsqueda de una educación para todos, sin barreras ni limitaciones. Al hacerlo, construiremos un futuro en el que la adquisición de conocimientos y habilidades sea una posibilidad accesible para cada persona, contribuyendo así al avance y prosperidad de la sociedad en su conjunto.

Por otra parte, el proyecto es bastante ambicioso para ser llevado a cabo por una sola persona en el tiempo limitado asignado. Por lo tanto es posible que con un equipo más extenso o con un plazo más amplio, se puedan abordar todas las mejoras mencionadas. Este análisis subraya la importancia de contar con recursos adecuados y tiempo suficiente al abordar iniciativas de esta magnitud para garantizar resultados más satisfactorios.

En adición, es crucial señalar que el objetivo principal del proyecto no pudo alcanzarse en su totalidad, principalmente debido a las limitaciones mencionadas anteriormente. El desarrollo del código de movimiento XYZ desde cero no resultó exitoso al intentar utilizar la cinemática inversa, lo que afectó negativamente la funcionalidad integral del sistema. Como consecuencia, diversas características esenciales no están operativas, ya que dependen directamente de este tipo de movimiento para evidenciar cambios físicos en el equipo.

En este contexto, se destaca la importancia de abordar proyectos de esta envergadura con recursos adecuados y un tiempo suficiente, ya que la complejidad técnica y las dificultades encontradas durante la implementación impactaron significativamente en la consecución de los objetivos planteados.

Asimismo, se reconoce que el proyecto aún tiene margen para mejoras significativas. Una revisión y pulido del diseño permitiría una interfaz más amigable para el usuario, acercándose de manera más efectiva a la operatividad que se busca lograr en un entorno real. Este enfoque, combinado con un equipo más extenso o un plazo más amplio, podría potenciar la capacidad del proyecto para abordar y superar los desafíos actuales, proporcionando resultados más satisfactorios en el futuro.

El código está anexado y el proyecto se encuentra alojado en el repositorio de GitHub \footnote{\url{https://github.com/Patricio1Labra/SimuladorCIMUBB}}, para que cualquiera pueda trabajar en el.