Este capítulo presenta el plan para capacitar y entrenar al equipo del proyecto. El objetivo es brindar las habilidades y conocimientos necesarios para lograr un uso exitoso del software.
\section{Plan de capacitación}

El plan de capacitación se centra en proporcionar a los participantes una comprensión integral del programa mediante la creación y consulta de una wiki alojada en GitHub \footnote{\url{https://github.com/Patricio1Labra/SimuladorCIMUBB/wiki}}. Esta plataforma servirá como un recurso centralizado y accesible para explorar todos los aspectos del programa, desde conceptos básicos hasta funciones avanzadas.

El contenido de la wiki abordará detalladamente el funcionamiento del programa, proporcionando documentación clara y ejemplos prácticos. Los participantes podrán acceder a tutoriales, guías paso a paso y recursos adicionales que facilitarán su comprensión y aplicación del programa en diversas situaciones.

Una vez que los participantes hayan adquirido una comprensión sólida a través de la wiki, el siguiente paso será una capacitación más especializada. En esta fase, un profesor capacitado asumirá la responsabilidad de guiar a los participantes a través del uso específico de los brazos robóticos asociados con el programa. Esta capacitación se enfocará en la operación práctica de los brazos robóticos, destacando las mejores prácticas, técnicas avanzadas y resolviendo cualquier pregunta o desafío que puedan enfrentar los participantes.

La combinación de la documentación en la wiki y la capacitación práctica sobre los brazos robóticos garantizará que los participantes adquieran un conocimiento completo y práctico del programa, lo que les permitirá utilizar eficazmente la tecnología en su aplicación real.