Este capítulo presenta el plan para capacitar y entrenar al equipo del proyecto. El objetivo es brindar las habilidades y conocimientos necesarios para lograr un uso exitoso del software.
\section{Plan de capacitación}

En el actual entorno educativo, la implementación de simuladores de laboratorio en Unity se ha vuelto esencial para ofrecer experiencias prácticas de manera eficiente y accesible. A continuación, se presenta la justificación de por qué no es necesario realizar capacitación extensa para utilizar este simulador, destacando sus características clave.

\begin{enumerate}
    \item \textbf{Acceso Inmediato:} El simulador ha sido diseñado para proporcionar un acceso inmediato a las prácticas virtuales de laboratorio. Los usuarios pueden sumergirse directamente en las actividades sin requerir una capacitación previa, facilitando una transición fluida a la experiencia virtual.

    \item \textbf{Interfaz Intuitiva:} La interfaz del simulador se ha estructurado de manera lógica, con controles fácilmente comprensibles. La disposición de las funciones es intuitiva, eliminando la necesidad de instrucciones detalladas y permitiendo que los usuarios se familiaricen rápidamente con el entorno.

    \item \textbf{Descripciones Informativas:} Cada opción dentro del simulador cuenta con descripciones claras y concisas. Estas proporcionan información detallada sobre las funciones de herramientas e instrumentos, permitiendo a los usuarios comprender su uso sin la necesidad de capacitación adicional.

    \item \textbf{Aprendizaje Visual:} Se ha dado énfasis a la naturaleza visual del aprendizaje en el simulador. Las representaciones gráficas de las acciones y resultados ayudan a los usuarios a comprender de manera rápida y efectiva los principios detrás de cada práctica de laboratorio.

    \item \textbf{Exploración Autónoma:} El simulador fomenta la exploración autónoma y el aprendizaje a través de la experiencia. Los usuarios pueden experimentar con las herramientas disponibles, recibiendo retroalimentación inmediata, lo que contribuye a un aprendizaje efectivo sin depender de sesiones formales de capacitación.

    \item \textbf{Minimización de Barreras de Aprendizaje:} La filosofía de diseño del simulador se centra en minimizar las barreras de aprendizaje. Esto se logra a través de la simplicidad en la operación, permitiendo a los usuarios aprovechar al máximo las funciones sin la necesidad de un extenso proceso de formación.
\end{enumerate}

Si bien, la aplicación fue desarrollada con la intención de hacerla lo mas accesible al usuario, no significa que el usuario este capacitado para entender el uso de los brazos robóticos.
Esto lleva a tener dos opciones:
\begin{enumerate}
    \item El usuario tiene un curso en el laboratorio: Si el usuario esta realizando un curso, la capacitación del uso de los brazos la otorga el profesor.
    \item El usuario no realiza un curso en el laboratorio: Si el usuario no esta realizando un curso en el laboratorio, puede realizar aprendizaje autónomo buscando información sobre el robot, pues la finalidad del simulador inicial es tener una alternativa al no tener acceso físico al laboratorio y no reemplazar la enseñanza otorgada por los profesores que dictan las clases en el.
\end{enumerate}