En este capítulo, se detalla el plan para implementar y poner en funcionamiento el software del brazo robótico.
\section{Plan de implantación}

La aplicación se ha concebido con la premisa de ser extremadamente accesible: basta con descargar, descomprimir y utilizar. Esta simplicidad se debe a que su objetivo es permitir que los estudiantes realicen operaciones con los equipos del laboratorio sin necesidad de estar físicamente presentes.
Se detalla las instrucciones a continuación

\section*{Instrucciones para el uso}

\subsection*{Descargar y Descomprimir}

\subsubsection*{Windows}

\begin{enumerate}[label=\arabic*.-]
    \item Descargar e instalar WinRAR u otro programa de su preferencia para descomprimir archivos ZIP en caso que el sistema no le permita de forma inmediata
    \item Descargar el archivo ZIP desde GitHub.
    \item Descomprimir el archivo ZIP haciendo clic derecho y seleccionando ''Extraer aquí''.
\end{enumerate}

\subsubsection*{Mac}

\begin{enumerate}[label=\arabic*.-]
    \item Utilizar la aplicación ''Unarchiver'' u otro programa de su preferencia para descomprimir archivos ZIP, o la línea de comandos.
    \item Descargar el archivo ZIP desde GitHub.
    \item En la línea de comandos: \verb|unzip nombre_del_archivo.zip|.
\end{enumerate}

\subsubsection*{Linux}

\begin{enumerate}[label=\arabic*.-]
    \item Verificar si el comando \verb|unzip| está instalado con: \verb|unzip --version|.
    \item Si no está instalado, instalarlo con el gestor de paquetes de tu distribución.
        \begin{itemize}
            \item Para Debian/Ubuntu: \verb|sudo apt-get install unzip|.
            \item Para Red Hat/Fedora: \verb|sudo yum install unzip|.
        \end{itemize}
    \item Descargar el archivo ZIP desde GitHub.
    \item Descomprimir el archivo ZIP con: \verb|unzip nombre_del_archivo.zip|.
\end{enumerate}

\subsection*{Ejecutar la Aplicación}

\subsubsection*{Windows}

\begin{itemize}
    \item Buscar el archivo ejecutable (con extensión .exe) dentro de la carpeta descomprimida.
    \item Hacer doble clic en el archivo ejecutable para iniciar la aplicación.
\end{itemize}

\subsubsection*{Mac}

\begin{itemize}
    \item Buscar el archivo ejecutable (con extensión .app o sin extensión) dentro de la carpeta descomprimida.
    \item Si es un archivo .app, hacer clic derecho y seleccionar ''Abrir'' para evitar problemas de seguridad.
    \item Si es un ejecutable, hacer doble clic para abrirlo.
\end{itemize}

\subsubsection*{Linux}

\begin{itemize}
    \item Abrir una terminal en la carpeta descomprimida.
    \item Verificar los permisos del archivo ejecutable con \verb|ls -l| y asegurarse de que sea ejecutable (\verb|chmod +x nombre_del_ejecutable| si es necesario).
    \item Ejecutar la aplicación desde la terminal con \verb|./nombre_del_ejecutable|.
\end{itemize}

Si bien algunas instrucciones pueden sonar como si el usuario necesitara tener un buen conocimiento sobre el equipo, la mayoría de los pasos implican funciones que normalmente uno realiza. Por ejemplo, enviar correos con archivos adjuntos comprimidos es una tarea común.