En la actualidad, el mundo se intenta recuperar de una pandemia, la cual afectó de manera significativamente negativa al área de educación \cite{Educacion}.


En la Universidad del Bío-Bío, el laboratorio de automatización llamado CIMUBB nace para llegar a preparar a los estudiantes para las nuevas tecnologías de automatización. En este edificio cuentan con distintos tipos de sistemas automatizados, o también se les nombra robots, de estos poseen diferentes tipos, pero en este proyecto se enfoca en el uso de los brazos robóticos de nombre SCORBOT, los cuales son encargados de la manipulación de objetos, su función es mover el objeto para ser depositado en un contenedor o bandeja para ser transportado en una cinta mecánica, o también dejar el objeto en otro robot. 

Gracias a este laboratorio, los alumnos pueden salir con el conocimiento para poder llegar a utilizar un sistema automatizado. Pero esta enseñanza fue mermada para cuando se mantuvo una cuarentena, ya que las clases se realizaron de forma on-line, el uso de estos robots fue demasiado limitado, el alumnos debía esperar que el profesor Luis Vera le agendara una videoconferencia, la cual tenia que esperar hasta que el profesor llegara a tener acceso al laboratorio, así dejando una comunicación entre el alumno y el computador central de los robots. Esto llevo a que el alumno el cual, aparte de tener que esperar, no tendría conocimiento directo sobre el uso del robot mas que verlo a través de una cámara.

El profesor se encuentra con la problemática de como lograr realizar la enseñanza de la materia practica en el laboratorio, pero sin tener el acceso a este mismo. Lo que lleva a una solución, que son las tecnologías de la simulación. Hoy existe un programa llamado Unity, en el cual sirve tanto para crear videojuegos como para crear simulaciones realistas que pueden ser utilizadas en cine, presentación de un coche o un edificio totalmente amueblado, etc.

Para este proyecto se mezcla la idea de usar la simulación y los videojuegos, pues gracias a estos últimos se logra realizar el movimiento. La idea en este proyecto es crear una simulación del laboratorio de automatización CIMUBB, en el cual se enfoca en darle vida a los brazos robóticos SCORBOT. Con esto lograr que las personas que deseen aprender sobre el funcionamiento de estos brazos robóticos, o también llegar a probar distintos experimentos en el brazo robótico, el limite de creatividad es el limite del brazo robótico.

%%Actualizar esto
En este escrito, se cuenta con ocho capítulos, los cuales están ordenados de la siguiente manera:
\begin{itemize}
    \item Capitulo 1: Este capítulo sirve como punto de partida para el trabajo, brindando una visión general del tema y su relevancia. Se establecen los objetivos de la investigación y se presenta la estructura del documento, preparando al lector para adentrarse en el análisis y exploración del tema en cuestión.
    \item Capitulo 2: Se exponen las razones y motivos detrás de la elección de este trabajo o proyecto en particular. Se describe la importancia del tema y su relevancia en el contexto actual, destacando las problemáticas o necesidades que aborda.
    \item Capitulo 3: Se proporciona una descripción de la empresa y de la problemática a tratar.
    \item Capitulo 4: Se establecen los objetivos y alcance del proyecto.
    \item Capitulo 5: Se detalla los requerimientos y funcionalidades que el software o proyecto debe cumplir.
    \item Capitulo 6: Se evalúa la viabilidad del proyecto.
    \item Capitulo 7: Se analiza los actores que participan en el software.
    \item Capitulo 8: Se describe el diseño del software que se desarrollará para resolver el problema identificado.
\end{itemize}