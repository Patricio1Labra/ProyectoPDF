La pandemia de COVID-19, que comenzó a principios de 2020, tuvo un impacto significativo en diversos sectores, y la educación no fue la excepción \cite{Educacion}. Con la propagación del virus, muchas instituciones educativas se vieron obligadas a cerrar temporalmente y adoptar modalidades de enseñanza en línea para garantizar la seguridad de estudiantes y personal.

En el caso específico de la Universidad del Bío-Bío, la transición a la educación en línea generó desafíos, especialmente en disciplinas prácticas que requerían acceso a laboratorios y equipamiento especializado. El laboratorio de automatización CIMUBB, que se centra en la preparación de estudiantes para las nuevas tecnologías de automatización, se vio afectado por estas limitaciones.

El uso de robots, en este caso, los brazos robóticos SCORBOT, se vio restringido durante la cuarentena, ya que los estudiantes no podían acceder físicamente al laboratorio. Esto resultó en una disminución en la experiencia práctica y en la interacción directa con los sistemas automatizados. Los desafíos adicionales surgieron al depender de videoconferencias para la enseñanza, lo que limitó la efectividad de la instrucción práctica.

La imposibilidad de acceder al laboratorio durante la pandemia llevó al profesor Luis Vera a buscar soluciones alternativas para continuar con la enseñanza práctica. Fue en este contexto que se exploraron las tecnologías de simulación, destacando Unity como una herramienta versátil que permitiría recrear virtualmente el laboratorio de automatización CIMUBB y, específicamente, dar vida a los brazos robóticos SCORBOT.

La simulación propuesta no solo aborda la limitación impuesta por la pandemia, sino que también amplía las posibilidades de aprendizaje. Al utilizar elementos de videojuegos, se ofrece a los estudiantes la oportunidad de interactuar de manera práctica con los brazos robóticos, incluso permitiéndoles realizar experimentos y desarrollar habilidades de control directo.

Esta iniciativa representa una respuesta innovadora a los desafíos educativos presentados por la pandemia, destacando cómo las tecnologías de simulación y la creatividad pueden superar barreras físicas y brindar experiencias de aprendizaje significativas, incluso en contextos adversos como los generados por el COVID-19.

\clearpage
En este escrito, se cuenta con catorce capítulos, los cuales están ordenados de la siguiente manera:
\begin{itemize}
    \item Capitulo 1 Introducción: Este capítulo sirve como punto de partida para el trabajo, brindando una visión general del tema y su relevancia. Se establecen los objetivos de la investigación y se presenta la estructura del documento, preparando al lector para adentrarse en el análisis y exploración del tema en cuestión.
    \item Capitulo 2 Justificación: Se exponen las razones y motivos detrás de la elección de este trabajo o proyecto en particular. Se describe la importancia del tema y su relevancia en el contexto actual, destacando las problemáticas o necesidades que aborda. Ademas se presentan soluciones alternativas para el problema.
    \item Capitulo 3 Definición de la institución: Se proporciona una descripción de la institución y de la problemática a tratar.
    \item Capitulo 4 Definición de proyecto: En este capitulo se establecen los objetivos, actividades, tecnologías y herramientas para el proyecto
    \item Capitulo 5 Especificación de requerimientos de software: Se detalla el alcance que tiene el proyecto, los requerimientos y funcionalidades que el software o proyecto debe cumplir.
    \item Capitulo 6 Factibilidad: Se evalúa la viabilidad del proyecto.
    \item Capitulo 7 Análisis: Se analiza los actores que participan en el software.
    \item Capitulo 8 Diseño: Se describe el diseño del software desarrollado para resolver el problema identificado.
    \item Capitulo 9 Código: Este capitulo se presenta y explica el código desarrollado con la finalidad de darle funcionalidad al programa desarrollado.
    \item Capitulo 10 Pruebas: Se presenta las pruebas realizadas del proyecto.
    \item Capitulo 11 Plan de capacitación y entrenamiento: Se detalla el plan para capacitar.
    \item Capitulo 12 Plan de implantación y puesta en marcha: Se presenta cómo se llevará a cabo la implementación y el inicio del proyecto.
    \item Capitulo 13 Trabajos Futuros: Se analizan ideas o propuestas para futuras mejoras o expansiones del proyecto.
    \item Capitulo 14 Conclusiones: Resumen de los hallazgos, conclusiones clave y posibles recomendaciones basadas en los resultados obtenidos durante el desarrollo del proyecto.
\end{itemize}