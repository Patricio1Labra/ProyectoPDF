En la actualidad, el mundo se intenta recuperar de una pandemia, la cual afectó de manera significativamente negativa al área de educación \cite{Educacion}.

En la Universidad del Bío-Bío, el laboratorio de automatización llamado CIMUBB nace para llegar a preparar a los estudiantes para las nuevas tecnologías de automatización. En este edificio cuentan con distintos tipos de sistemas automatizados, o también se les nombra robots, de estos poseen diferentes tipos, pero en este proyecto se enfoca en el uso de los brazos robóticos de nombre SCORBOT, los cuales son encargados de la manipulación de objetos, su función es mover el objeto para ser depositado en un contenedor o bandeja para ser transportado en una cinta mecánica, o también dejar el objeto en otro robot. 

Gracias a este laboratorio, los alumnos pueden salir con el conocimiento para poder llegar a utilizar un sistema automatizado. Pero esta enseñanza fue mermada para cuando se mantuvo una cuarentena, ya que las clases se realizaron de forma on-line, el uso de estos robots fue demasiado limitado, el alumnos debía esperar que el profesor Luis Vera le agendara una videoconferencia, la cual tenia que esperar hasta que el profesor llegara a tener acceso al laboratorio, así dejando una comunicación entre el alumno y el computador central de los robots. Esto llevo a que el alumno el cual, aparte de tener que esperar, no tendría conocimiento directo sobre el uso del robot mas que verlo a través de una cámara.

El profesor se encuentra con la problemática de como lograr realizar la enseñanza de la materia practica en el laboratorio, pero sin tener el acceso a este mismo. Todo esto debido a las limitaciones que fueron impuestas con la finalidad de evitar contagios, pues para ese momento, las clases debían ser telemáticas.

Este problema necesitaba una solución, una respuesta para esta problemática son las tecnologías de la simulación. Hoy existe un programa llamado Unity, en el cual sirve tanto para crear videojuegos como para crear simulaciones realistas que pueden ser utilizadas en cine, presentación de un coche o un edificio totalmente amueblado, etc.

Para este proyecto se mezcla la idea de usar la simulación y los videojuegos, pues gracias a estos últimos se logra realizar el movimiento controlado por el usuario. La idea en este proyecto es crear una simulación del laboratorio de automatización CIMUBB, en el cual se enfoca en darle vida a los brazos robóticos SCORBOT. Con esto lograr que las personas que deseen aprender sobre el funcionamiento de estos brazos robóticos, o también llegar a probar distintos experimentos en el brazo robótico, el limite de creatividad es el limite del brazo robótico.

En este escrito, se cuenta con catorce capítulos, los cuales están ordenados de la siguiente manera:
\begin{itemize}
    \item Capitulo 1 Introducción: Este capítulo sirve como punto de partida para el trabajo, brindando una visión general del tema y su relevancia. Se establecen los objetivos de la investigación y se presenta la estructura del documento, preparando al lector para adentrarse en el análisis y exploración del tema en cuestión.
    \item Capitulo 2 Justificación: Se exponen las razones y motivos detrás de la elección de este trabajo o proyecto en particular. Se describe la importancia del tema y su relevancia en el contexto actual, destacando las problemáticas o necesidades que aborda. Ademas se presentan soluciones alternativas para el problema.
    \item Capitulo 3 Definición de la institución: Se proporciona una descripción de la institución y de la problemática a tratar.
    \item Capitulo 4 Definición de proyecto: En este capitulo se establecen los objetivos, actividades, tecnologías y herramientas para el proyecto
    \item Capitulo 5 Especificación de requerimientos de software: Se detalla el alcance que tiene el proyecto, los requerimientos y funcionalidades que el software o proyecto debe cumplir.
    \item Capitulo 6 Factibilidad: Se evalúa la viabilidad del proyecto.
    \item Capitulo 7 Análisis: Se analiza los actores que participan en el software.
    \item Capitulo 8 Diseño: Se describe el diseño del software desarrollado para resolver el problema identificado.
    \item Capitulo 9 Código: Este capitulo se presenta y explica el código desarrollado con la finalidad de darle funcionalidad al programa desarrollado.
    \item Capitulo 10 Pruebas: Se presenta las pruebas realizadas del proyecto.
    \item Capitulo 11 Plan de capacitación y entrenamiento: Se detalla el plan para capacitar.
    \item Capitulo 12 Plan de implantación y puesta en marcha: Se presenta cómo se llevará a cabo la implementación y el inicio del proyecto.
    \item Capitulo 13 Trabajos Futuros: Se analizan ideas o propuestas para futuras mejoras o expansiones del proyecto.
    \item Capitulo 14 Conclusiones: Resumen de los hallazgos, conclusiones clave y posibles recomendaciones basadas en los resultados obtenidos durante el desarrollo del proyecto.
\end{itemize}